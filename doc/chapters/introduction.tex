\chapter{Introduction}
\label{chapter:introduction}

Skin cancer is a major public health problem, with over 5 million newly diagnosed cases in the United States each year. Melanoma is the deadliest form of skin cancer, responsible for over 9,000 deaths each year.

As pigmented lesions occurring on the surface of the skin, melanoma is amenable to early detection by expert visual inspection. It is also amenable to automated detection with image analysis. Given the widespread availability of high-resolution cameras, algorithms that can improve our ability to screen and detect troublesome lesions can be of great value. As a result, many centers have begun their own research efforts on automated analysis. However, a centralized, coordinated, and comparative effort across institutions has yet to be implemented.

Dermoscopy is an imaging technique that eliminates the surface reflection of skin. By removing surface reflection, visualization of deeper levels of skin is enhanced. Prior research has shown that when used by expert dermatologists, dermoscopy provides improved diagnostic accuracy, in comparison to standard photography. As inexpensive consumer dermatoscope attachments for smart phones are beginning to reach the market, the opportunity for automated dermoscopic assessment algorithms to positively influence patient care increases.

The International Skin Imaging Collaboration (ISIC) is an international effort to improve melanoma diagnosis, sponsored by the International Society for Digital Imaging of the Skin (ISDIS). The ISIC Archive contains the largest publicly available collection of quality controlled dermoscopic images of skin lesions.

Presently, the ISIC Archive contains over 13,000 dermoscopic images, which were collected from leading clinical centers internationally and acquired from a variety of devices within each center. Broad and international participation in image contribution is designed to insure a representative clinically relevant sample. Initial data sets derive primarily from the United States, and the ISIC has ongoing commitments and contributions of additional data sets from international collaborators.

All incoming images to the ISIC Archive are screened for both privacy and quality assurance. Most images have associated clinical metadata, which has been vetted by recognized melanoma experts. A subset of the images have undergone annotation and markup by recognized skin cancer experts. These markups include dermoscopic features (i.e., global and focal morphologic elements in the image known to discriminate between types of skin lesions).

The software infrastructure of the ISIC Archive is built using the open-source Girder platform, and the source code for the Archive itself is freely available on GitHub. All the data and annotations presently within the ISIC Archive are licensed for free public download and use, under the CC-0 license.

😂
